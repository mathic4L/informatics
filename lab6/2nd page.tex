\documentclass{article}
\usepackage{cmap} % улучшенный поиск русских слов в pdf
\usepackage[T2A]{fontenc}
\usepackage[utf8]{inputenc}
\usepackage{multicol}
\usepackage[english, russian] {babel}
\usepackage{amsfonts,ams symb}
\usepackage{graphicx} 
\graphicspath{ {images/} }
\usepackage{float} 
\usepackage{makecell}
\usepackage{footnote}
\usepackage{rotating}
\usepackage{starfootnote}
\usepackage{style}
\begin{document}
\begin{multicols}{2}
Хочется сразу написать
\[
\sqrt{2}=\frac{1}{
    2 + \frac{1}{
        2 + \frac{1}{
            2 + \frac{1}{
                2 + \cdot
            }
        }
    }
}=
\]
\[
=[1; 2, 2, 2, 2, ...],
\]
т. е. представить $\sqrt{2}$ в виде \textit{беско- нечной цепной дроби.} 
Но здесь нужна крайняя осторожность: мы встрети лись с новым понятием «бесконечная десятичная дробь», но не знаем, что это такое.
Легко понять только, что каждому положительному иррацио- нальному числу 
с\kern1emо\kern1emо\kern1emт\kern1emв\kern1emе\kern1emт\kern1emс\kern1emт\kern1emв\kern1emу\kern1emе\kern1emт 
вполне определенная бесконеч ная последовательность
\[
[a_0; a_1, a_2, ...],
\]
где $a_{0}$ --- целое не отрицательное, а все $a_i$, с номером i$\geq$1 --- натуральные числа. Во всем относящемся сюда мы разберемся только позже
\starfootnote{Этот вопрос будет полностью рассмотрен в одном из следующих номеров журнала}
, а пока удовлетворимся тем, что нам понятно: как по положительному числу $\alpha$ по- строить его формальное разложение 
\[
a \sim [a_0, a_1, a_2, ...]
\]
в конечную или бесконечную цепную дробь
\starfootnoteDouble{``$\sim$''  --- знак соответствия. Мы боимся поставить знак равенства, пока не установлен смысл символа, стоящего в правой части.}
.

5. П\kern1emо\kern1emд\kern1emх\kern1emо\kern1emд\kern1emя\kern1emщ\kern1emи\kern1emе д\kern1emр\kern1emо\kern1emб\kern1emи. 
Цепную дробь можно оборвать, удер жав элементы $a_o$, $a_1$, ..., $a_n$ и отбро сив $a_{n+1}$... . Полученное таким обра зом число называется 
n-й \textit{подходя- щей дробью и обозначается} $\frac{p_n}{q_n}$ . В част- ности, при n=0 имеем \textit{нулевую код- ходящую дробь}
$\frac{p_n}{q_n}=\frac{a_0}{1}$

Мы увидим, что, чем меньше n, тем подходящая дробь проще (т. е. имеет меньший знаменатель). 
В то же время она может использоваться как приближенное значение цепной дроби.
\end{multicols}
\end{document}


